\documentclass[sigconf]{acmart}

%\immediate\write18{wget -Nq  https://raw.githubusercontent.com/bigdata-i523/sample-hid000/master/paper1/i523.tex}
%\input{i523}

\usepackage{graphicx}
\usepackage{hyperref}

\usepackage{endfloat}
\renewcommand{\efloatseparator}{\mbox{}} % no new page between figures

\usepackage{booktabs} % For formal tables

\settopmatter{printacmref=false} % Removes citation information below abstract
\renewcommand\footnotetextcopyrightpermission[1]{} % removes footnote with conference information in first column
\pagestyle{plain} % removes running headers


\begin{document}


\title{Big Data for Edge Computing}


\author{Ben Trovato}
\orcid{1234-5678-9012}
\affiliation{%
  \institution{Institute for Clarity in Documentation}
  \streetaddress{P.O. Box 1212}
  \city{Dublin} 
  \state{Ohio} 
  \postcode{43017-6221}
}
\email{trovato@corporation.com}

\author{G.K.M. Tobin}
\affiliation{%
  \institution{Institute for Clarity in Documentation}
  \streetaddress{P.O. Box 1212}
  \city{Dublin} 
  \state{Ohio} 
  \postcode{43017-6221}
}
\email{webmaster@marysville-ohio.com}

\author{Gregor von Laszewski}
\affiliation{%
  \institution{Indiana University}
  \streetaddress{Smith Research Center}
  \city{Bloomington} 
  \state{IN} 
  \postcode{47408}
  \country{USA}}
\email{laszewski@gmail.com}


% The default list of authors is too long for headers}
\renewcommand{\shortauthors}{G. v. Laszewski}


\begin{abstract}
This paper provides a sample of a \LaTeX\ document which conforms,
somewhat loosely, to the formatting guidelines for
ACM SIG Proceedings.
\end{abstract}

\keywords{Big Data, Edge Computing i523}


\maketitle

\section{Introduction}

Put here an introduction about your topic. 
We just need one sample refernce so the paper compiles in LaTeX so we
put it here \cite{editor00}.


\section{From PDF}

At the core of Big Data is a challenge.  A challenge of exploration
“of the complexities inherently trapped in data, business, and
problem-solving systems'' (Cao, 2017) \cite{??} which is by definition, ``Big
Data''.  Imagine a world where your business decisions relate to data
sources that range from a flat file from a third-party vendor to
millions of internal data records every day, nearly every hour.  Add
to this, some data sources might ``round up'' the data, while others
relate the data (traffic) to a different geographic standard then
others.  So it is in the world of mobility network traffic.  Mobility
network traffic providers generate CDR’s \cite{??} (Call Detail Records) every
time a device establishes a connection.  These CDRs provide details
about the connection – cell site locations, length of call and device
information.  It is from these records that the network providers
gather, ``clean'' if need be, consolidate and extrapolate the needed
information to bill the customer.

As shown in Figure \ref{F:graph} ....


While CDR’s tell us a great deal, there is much that they don’t tell a
provider.  For example, by the time the millions of records are
consolidated to generate files that are more manageable, data details
can be lost.  Therefore, other data sources are used, like data from
the network, which provide precise traffic metrics.  Adding to the
challenge, companies like Verizon and AT\&T are changing to unlimited
plans, offering package deals with video services and even offering
free traffic based on cell phone apps (HBO for free on your device) –
This data set is much different than simply looking at network or CDR
traffic. That is, we need to look at the bits and bytes. Additionally,
the customer landscape has changed from traditional post-paid
customers to those that pre-paid or are sold as wholesale or the IoT
customers.  With IoT, lots of projections abound and here is one:
``roughly 23 billion active IoT devices by the year 2019….spending on
enterprise IoT products and services will reach
\$255 billion globally by 2019, up from \$46.2 billion this year. ''
(Schofield, 2015) Also network providers have learned that nothing
puts more traffic on the network like video.  Video based apps, like
Facebook and You Tube directly impact network traffic.  The impact of
apps on the mobility network is significant with no end in site:
``…when it comes to reaching consumers in masse…the market has
confirmed what we’ve known all along – that we are all building and
investing into a platform that can reach heights we may have never
seen before.  That, to me, is ``The WhatsApp Effect, and there is no
turning back now.'' (Shah, 2014) \cite{??}

For mobility network providers, what is the Big Data challenge here?
Providers already have access to network traffic data, along with data
around traffic above the (OSI) network layer to provide some insights
into traffic types; web browsing traffic, VoIP, video, and even some
data around traffic related to apps.  The challenge for Big Data is to
take all of this data and give network providers accurate readings on
- customer behavior!  Can it be done?  I believe, with the use of data
holistically and with data-driven discovery, it can.  In order for
this to be successful, you have to have a solid understanding of the
data itself and substantive data storage capabilities, like data
lakes.  A holistic view of the data is to include all the data
sources; network data, traffic type data, app level data interrelated
and connected hierarchically, so that when you see a jump in the
network traffic, you trace the traffic type and app level, which can
then lead to accurate deductions to explain the, aberration, one such
as, ``The Ice Bucket Challenge'', an innocuous social experiment played
out on Facebook that demanded a tremendous amount of network capacity.
This comprehensive, holistic approach is the only way to paint an
accurate picture of user behavior, taming ``Big Data'' into a beast that
can be interpreted.  And as a result, helping understand – customer
behavior.  At this point we have gained wisdom and data-driven
discovery that can be applied to the network itself.  Impacting the
bottom line.



\begin{figure}
\includegraphics[width=\columnwidth]{images/graph}
\caption{A sample black and white graphic. \cite{??}}
\label{F:graph}
\end{figure}



\section{Conclusion}

Put here an conclusion. Conlcusions and abstracts must not have any
citations in the section.


\begin{acks}

  The authors would like to thank Dr. Gregor von Laszewski for his
  support and suggestions to write this paper.

\end{acks}

\bibliographystyle{ACM-Reference-Format}
\bibliography{report} 

\end{document}
